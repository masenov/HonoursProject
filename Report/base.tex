% to choose your degree
% please un-comment just one of the following

\documentclass[bsc,frontabs,singlespacing,parskip,deptreport,logo]{infthesis}     % for BSc, BEng etc.
% \documentclass[minf,frontabs,twoside,singlespacing,parskip,deptreport]{infthesis}  % for MInf


\usepackage{tikz}
\usetikzlibrary{timeline}


\usepackage{graphicx}
\usepackage{url}
\usepackage{amsmath}
\usepackage{caption}
\usepackage{subcaption}
\usepackage{gensymb}
%\usepackage[capposition=top]{floatrow}


\graphicspath{ {Pictures/background/}{Pictures/models/}{Pictures/elastica/}{Pictures/illusions/} }



\newcommand{\plottwofigures}[3]{
\begin{figure}[]
\caption{}
\begin{subfigure}{.5\textwidth}
  \caption{}
  \centering
  \includegraphics[width=.8\linewidth]{#1}
\end{subfigure}%
\begin{subfigure}{.5\textwidth}
  \caption{}
  \centering
  \includegraphics[width=.8\linewidth]{#2}
\end{subfigure}
\label{#1}
\floatfoot{#3}
\end{figure}
}


\newcommand{\plottwofiguresS}[4]{
\begin{figure}[]
\caption{}
\begin{subfigure}{.5\textwidth}
  \caption{}
  \centering
  \includegraphics[width=#4\linewidth]{#1}
\end{subfigure}%
\begin{subfigure}{.5\textwidth}
  \caption{}
  \centering
  \includegraphics[width=#4\linewidth]{#2}
\end{subfigure}
\label{#1}
\floatfoot{#3}
\end{figure}
}


\newcommand{\plotfigure}[2]{
\begin{figure}[]
\caption{}
\centering
\includegraphics[width=.8\linewidth]{#1}
\label{#1}
\begin{center}
\floatfoot{#2}
\end{center}
\end{figure}
}


\newcommand{\plotfigureS}[3]{
\begin{figure}[]
\caption{}
\centering
\includegraphics[width=#3\linewidth]{#1}
\label{#1}
\floatfoot{#2}
\end{figure}
}

\newcommand{\plotthreefigures}[4]{
\begin{figure}[]
\caption{}
\begin{subfigure}{.5\textwidth}
  \caption{}
  \centering
  \includegraphics[width=1\linewidth]{#1}
\end{subfigure}%
\begin{subfigure}{.5\textwidth}
  \caption{}
  \centering
  \includegraphics[width=1\linewidth]{#2}
\end{subfigure}
\begin{subfigure}{.5\textwidth}
  \caption{}
  \centering
  \includegraphics[width=1\linewidth]{#3}
\end{subfigure}%
\label{#1}
\floatfoot{#4}
\end{figure}
}

\newcommand{\plotfigures}[5]{
\begin{figure}[]
\caption{}
\begin{subfigure}{.5\textwidth}
  \caption{}
  \centering
  \includegraphics[width=1\linewidth]{#1}
\end{subfigure}%
\begin{subfigure}{.5\textwidth}
  \caption{}
  \centering
  \includegraphics[width=1\linewidth]{#2}
\end{subfigure}
\begin{subfigure}{.5\textwidth}
  \caption{}
  \centering
  \includegraphics[width=1\linewidth]{#3}
\end{subfigure}%
\begin{subfigure}{.5\textwidth}
  \caption{}
  \centering
  \includegraphics[width=.1\linewidth]{#4}
\end{subfigure}
\label{#1}
\floatfoot{#5}
\end{figure}
}

\newcommand{\plotfiguresix}[8]{
\begin{figure}[]
\caption{}
\begin{subfigure}{.5\textwidth}
  \caption{}
  \centering
  \includegraphics[width=#8\linewidth]{#1}
\end{subfigure}%
\begin{subfigure}{.5\textwidth}
  \caption{}
  \centering
  \includegraphics[width=#8\linewidth]{#2}
\end{subfigure}
\begin{subfigure}{.5\textwidth}
  \caption{}
  \centering
  \includegraphics[width=#8\linewidth]{#3}
\end{subfigure}%
\begin{subfigure}{.5\textwidth}
  \caption{}
  \centering
  \includegraphics[width=#8\linewidth]{#4}
\end{subfigure}
\begin{subfigure}{.5\textwidth}
  \caption{}
  \centering
  \includegraphics[width=#8\linewidth]{#5}
\end{subfigure}%
\begin{subfigure}{.5\textwidth}
  \caption{}
  \centering
  \includegraphics[width=#8\linewidth]{#6}
\end{subfigure}
\label{#1}
\floatfoot{#7}
\end{figure}
}

\newcommand{\plotfiguresixn}[7]{
\begin{figure}[]
\caption{}
\begin{subfigure}{.5\textwidth}
  \centering
  \includegraphics[width=.8\linewidth]{#1}
\end{subfigure}%
\begin{subfigure}{.5\textwidth}
  \centering
  \includegraphics[width=.8\linewidth]{#2}
\end{subfigure}
\begin{subfigure}{.5\textwidth}
  \centering
  \includegraphics[width=.8\linewidth]{#3}
\end{subfigure}%
\begin{subfigure}{.5\textwidth}
  \centering
  \includegraphics[width=.8\linewidth]{#4}
\end{subfigure}
\begin{subfigure}{.5\textwidth}
  \centering
  \includegraphics[width=.8\linewidth]{#5}
\end{subfigure}%
\begin{subfigure}{.5\textwidth}
  \centering
  \includegraphics[width=.8\linewidth]{#6}
\end{subfigure}
\label{#1}
\floatfoot{#7}
\end{figure}
}

\newcommand{\bt}[1]{
\textbf{#1}
}

\newcommand{\eq}[1]{
\begin{equation}
  {#1}
\end{equation}
}
