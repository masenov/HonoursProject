% to choose your degree
% please un-comment just one of the following

% \documentclass[bsc,frontabs,twoside,singlespacing,parskip,deptreport,logo]{styles/infthesis}     % for BSc, BEng etc.
% \documentclass[minf,frontabs,twoside,singlespacing,parskip,deptreport]{infthesis}  % for MInf
\documentclass[bsc, twoside, parskip, logo, notimes, normalheadings]{styles/infthesis}


\usepackage{tikz}
\usetikzlibrary{timeline}


\usepackage{graphicx}
\usepackage{url}
\usepackage{amsmath}
\usepackage{caption}
\usepackage{subcaption}
\usepackage{gensymb}
\usepackage[capposition=bottom]{floatrow}


\graphicspath{ {Pictures/Introduction/} {Pictures/background/}{Pictures/models/}{Pictures/elasticap/}{Pictures/illusions/}{Pictures/experiments/rnn/} {Pictures/experiments/exp1_fig2/} {Pictures/experiments/fig5/} {Pictures/experiments/fig5/figB/} {Pictures/experiments/fig5/figC/} {Pictures/experiments/fig5/figD/} {Pictures/experiments/fig5/figF/} {Pictures/experiments/} {Pictures/experiments/elastica2/} {Pictures/experiments/fig5/figF/a15/} {Pictures/experiments/fig5/figF/a30/} {Pictures/experiments/fig5/figF/a60/} {Pictures/experiments/basics/} {Pictures/experiments/fig7/} }



\newcommand{\plottwofigures}[4]{
\begin{figure}[H]
\centering
\begin{subfigure}{.5\textwidth}
  \centering
  \includegraphics[width=.8\linewidth]{#1}
  \caption{}
\end{subfigure}%
\begin{subfigure}{.5\textwidth}
  \centering
  \includegraphics[width=.8\linewidth]{#2}
  \caption{}
\end{subfigure}
\caption[#4]{#3}
\label{#1}
\end{figure}
}


\newcommand{\plottwofiguresS}[5]{
\begin{figure}[H]
\centering
\begin{subfigure}{.5\textwidth}
  \caption{}
  \centering
  \includegraphics[width=#5\linewidth]{#1}
\end{subfigure}%
\begin{subfigure}{.5\textwidth}
  \caption{}
  \centering
  \includegraphics[width=#5\linewidth]{#2}
\end{subfigure}
\caption[#4]{#3}
\label{#1}
\end{figure}
}


\newcommand{\plotfigure}[3]{
\begin{figure}[H]
\centering
\includegraphics[width=.8\linewidth]{#1}
\caption[#3]{#2}
\label{#1}
\end{figure}
}


\newcommand{\plotfigureS}[4]{
\begin{figure}[H]
\centering
\includegraphics[width=#4\linewidth]{#1}
\caption[#3]{#2}
\label{#1}
\end{figure}
}

\newcommand{\plotthreefigures}[6]{
\begin{figure}[H]
\begin{subfigure}{.33\textwidth}
  \centering
  \includegraphics[width=#6\linewidth]{#1}
  \caption{}
\end{subfigure}%
\hfill
\begin{subfigure}{.33\textwidth}
  \centering
  \includegraphics[width=#6\linewidth]{#2}
  \caption{}
\end{subfigure}
\hfill
\begin{subfigure}{.33\textwidth}
  \centering
  \includegraphics[width=#6\linewidth]{#3}
  \caption{}
\end{subfigure}%
\caption[#5]{#4}
\label{#1}
\end{figure}
}

\newcommand{\plotfourfigures}[6]{
\begin{figure}[H]
\begin{subfigure}{.5\textwidth}
  \centering
  \includegraphics[width=#6\linewidth]{#1}
  \caption{}
\end{subfigure}%
\hfill
\begin{subfigure}{.5\textwidth}
  \centering
  \includegraphics[width=#6\linewidth]{#2}
  \caption{}
\end{subfigure}
\hfill
\begin{subfigure}{.5\textwidth}
  \centering
  \includegraphics[width=#6\linewidth]{#3}
  \caption{}
\end{subfigure}%
\hfill
\begin{subfigure}{.5\textwidth}
  \centering
  \includegraphics[width=#6\linewidth]{#4}
  \caption{}
\end{subfigure}%
\caption{#5}
\label{#1}
\end{figure}
}

\newcommand{\plotfigures}[5]{
\begin{figure}[H]
\caption{}
\begin{subfigure}{.5\textwidth}
  \caption{}
  \centering
  \includegraphics[width=1\linewidth]{#1}
\end{subfigure}%
\begin{subfigure}{.5\textwidth}
  \caption{}
  \centering
  \includegraphics[width=1\linewidth]{#2}
\end{subfigure}
\begin{subfigure}{.5\textwidth}
  \caption{}
  \centering
  \includegraphics[width=1\linewidth]{#3}
\end{subfigure}%
\begin{subfigure}{.5\textwidth}
  \caption{}
  \centering
  \includegraphics[width=1\linewidth]{#4}
\end{subfigure}
\label{#1}
\floatfoot{#5}
\end{figure}
}

\newcommand{\plotfiguresix}[9]{
\begin{figure}[H]
\centering
\begin{subfigure}[b]{#8\textwidth}
  \centering
  \includegraphics[width=\textwidth]{#1}
  \caption{}
\end{subfigure}
\hfill
\begin{subfigure}[b]{#8\textwidth}
  \centering
  \includegraphics[width=\textwidth]{#2}
  \caption{}
\end{subfigure}
\hfill
\begin{subfigure}[b]{#8\textwidth}
  \centering
  \includegraphics[width=\textwidth]{#3}
  \caption{}
\end{subfigure}
\hfill
\begin{subfigure}[b]{#8\textwidth}
  \centering
  \includegraphics[width=\textwidth]{#4}
  \caption{}
\end{subfigure}
\hfill
\begin{subfigure}[b]{#8\textwidth}
  \centering
  \includegraphics[width=\textwidth]{#5}
  \caption{}
\end{subfigure}
\hfill
\begin{subfigure}[b]{#8\textwidth}
  \centering
  \includegraphics[width=\textwidth]{#6}
  \caption{}
\end{subfigure}
\caption[#9]{#7}
\label{#1}
\end{figure}
}

\newcommand{\plotfiguresixn}[7]{
\begin{figure}[H]
\caption{}
\begin{subfigure}{.5\textwidth}
  \centering
  \includegraphics[width=.8\linewidth]{#1}
\end{subfigure}%
\begin{subfigure}{.5\textwidth}
  \centering
  \includegraphics[width=.8\linewidth]{#2}
\end{subfigure}
\begin{subfigure}{.5\textwidth}
  \centering
  \includegraphics[width=.8\linewidth]{#3}
\end{subfigure}%
\begin{subfigure}{.5\textwidth}
  \centering
  \includegraphics[width=.8\linewidth]{#4}
\end{subfigure}
\begin{subfigure}{.5\textwidth}
  \centering
  \includegraphics[width=.8\linewidth]{#5}
\end{subfigure}%
\begin{subfigure}{.5\textwidth}
  \centering
  \includegraphics[width=.8\linewidth]{#6}
\end{subfigure}
\label{#1}
\floatfoot{#7}
\end{figure}
}

\newcommand{\bt}[1]{
\textbf{#1}
}

\newcommand{\eq}[1]{
\begin{equation}
  {#1}
\end{equation}
}

\newcommand{\eql}[2]{
\begin{equation} \label{#2}
  {#1}
\end{equation}
}
