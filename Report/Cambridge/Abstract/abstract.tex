% ************************** Thesis Abstract *****************************
% Use `abstract' as an option in the document class to print only the titlepage and the abstract.
\begin{abstract}
We experience different tilt illusions and pop out effects. This is often explained by the line's representation in the visual cortex. A group of neurons is responsible for detecting different orientations in different parts of our visual field. A neuron tuned for a specific orientation spikes more, when its preferred orientation is presented, and less the more different the orientations is, forming an bell shaped curve activity. We have multiple neurons for every part of our visual field, tuned for different orientations. Combining their responses we get an accurate response of the actual orientation, despite their noisy spiking.

However neurons responsible for different part of our visual field are also connected to each other. This modulation explains some of the effect we are experiencing, orientations in our visual fields are influenced by nearby ones. However those interaction have only been explored statically, calculating the modulation only ones. In a full dynamic a change in an orientation, can lead to a change in another orientation, and so on. It's unclear if existing models change in a dynamic setting, does the system settles down to a stable state, or it keeps fluctuating.

In this report, we explore an already existing passive model of the above mentioned neurons and build a dynamic model. We explore the differences between the two to check if in a dynamic setting, we still experience the same effects. 
\end{abstract}
